%\setcounter{page}{4}
\addchap{Введение}

Современная компьютерная графика широко используется в наши дни. Она помогает людям решить ряд задач, например, создание спецэффектов в кинематографе, разработка визуального контента в играх или анимации, моделирование различных установок. Для всего этого очень важно делать получаемое изображение максимально приближенным к реальному.

В повседневной жизни мы часто пользуемся таким предметом как зеркало. Для нас важно, чтобы то, что мы в нем видим, было четким и без искажения. Качество этого изображения напрямую зависит от покрытия зеркала. Его шероховатости влияют на отражение света от поверхности, а значит и на полученное мнимое изображение.

Целью курсовой работы является построение реалистичного изображения сцены, на которой располагаются трехмерные объекты и визуализируется их отражение в зеркале, свойства которого можно изменять.

%Целью практики является подготовка необходимой базы для разработки программы, которая позволит визуализировать отражение в плоском зеркале и изучить влияние диффузной составляющей отражающего покрытия на полученное изображение. 

Для достижения поставленной цели, необходимо решить следующие задачи:
\begin{itemize}
	\item рассмотреть и выбрать алгоритмы трехмерной графики для визуализации объектов и построения их отражения в зеркале;
	\item изучить работу выбранных алгоритмов;
	\item спроектировать архитектуру программы и её интерфейс;
	\item реализовать выбранные алгоритмы и структуры данных.
\end{itemize}

Итогом работы станет программа, демонстрирующая отражение объектов в плоском зеркале и позволяющая менять диффузную составляющую его покрытия. Также должна быть предусмотрена возможность перемещения и поворота камеры, изменения положения и интенсивности источника освещения.