\addchap{Заключение}

В ходе выполнения данной работы были выполнены следующие задачи:
\begin{itemize}
	\item рассмотрены и выбраны алгоритмы трехмерной графики для визуализации объектов и построения их отражения в зеркале;
	\item изучена работа выбранных алгоритмов;
	\item спроектирована архитектуру программы и её интерфейс;
	\item реализованы выбранные алгоритмы и структуры данных.
\end{itemize}

В результате был реализован программный продукт для генерации реалистичного изображения сцены с зеркалом и трехмерным геометрическими объектами. Используя графический интерфейс, можно менять набор объектов сцены, характеристики источника света, положение и направление камеры.

Была проведена апробация работы перемещения камеры, изменения источника освещения, а также изменения диффузной составляющей отражения в плоском зеркале. В ходе проведенного эксперимента было установлено, что максимальная производительность достигается при реализации на 10 потоках, а выигрыш в данном случае достигает 55\%. 